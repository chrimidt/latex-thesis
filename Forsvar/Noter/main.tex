\documentclass[a4paper,12pt]{article}
% Packages and setup code for notes
\usepackage{graphicx}
\usepackage{subcaption}
\usepackage{caption}
\captionsetup{font=small}
\usepackage{fancyhdr}
\usepackage{geometry}
\usepackage{xcolor}
\usepackage{amsmath}
\usepackage{amssymb}
\usepackage{mathtools}
\usepackage[breaklinks]{hyperref}
\usepackage{url}
\usepackage{tabularx}
\usepackage{multirow}
\usepackage{booktabs}   % For \toprule, \midrule, \bottomrule
\usepackage{float}      % For forcing table placement
\usepackage{titlesec}   % Custom section formatting
\usepackage{enumitem}   % Control over lists

% Page geometry for notes
\geometry{
    a4paper,
    left=2.5cm,
    right=2.5cm,
    top=2.5cm,
    bottom=2.5cm,
    headheight=14pt
}

% Fancy headers for notes
\pagestyle{fancy}
\fancyhf{} % Clear existing header/footer settings
\fancyhead[L]{\small Notes for Defense} % Custom header
\setlength{\headheight}{14.49998pt}
\fancyhead[R]{\thepage} % Page number on the right
\fancyfoot[C]{\scriptsize Prepared by: \AuthorName}

% Highlighting and notes commands
\newcommand{\important}[1]{\textcolor{red}{\textbf{#1}}} % Important notes
\newcommand{\note}[1]{\textcolor{blue}{\textit{#1}}}      % General notes
\newcommand{\todo}[1]{\textcolor{orange}{TODO: #1}}       % TODOs

% Section and subsection formatting for clarity
\titleformat{\section}{\large\bfseries}{\thesection}{1em}{}
\titleformat{\subsection}{\normalsize\bfseries}{\thesubsection}{1em}{}

% Custom bullet points for lists
\renewcommand{\labelitemi}{$\bullet$}
\renewcommand{\labelitemii}{$\circ$}

% Hyperlink settings
\hypersetup{
    colorlinks=true,
    linkcolor=blue,
    filecolor=magenta,
    urlcolor=cyan,
    pdfborder={0 0 0},
}

% Custom commands
\newcommand{\AuthorName}{Christine Annelise Midtgaard} % Your name
\newcommand{\SupervisorName}{Kim Bjerge}              % Supervisor's name

% Spacing adjustments for notes
\setlength{\parskip}{0.5em}   % Slight space between paragraphs
\setlength\parindent{0pt}     % No paragraph indentation


\begin{document}

\title{Thesis Defense Preparation: \\ \large "Deep Learning Techniques for Long-Tailed Image Classification: A Comparative Study of Loss Designs and Model Architectures"}
\author{Christine Annelise Midtgaard}
\date{\today}
\maketitle

\tableofcontents

\newpage

\section{Introduction}
This section focuses on the overall preparation strategy for the defense.

\subsection{Purpose of the Defense}
\begin{itemize}
    \item Summarize the thesis content and key findings.
    \item Answer committee questions confidently and succinctly.
    \item Demonstrate the impact and relevance of the work.
\end{itemize}

\subsection{Preparation Workflow}
\begin{enumerate}
    \item Review the thesis and highlight key points.
    \item Develop clear slides for the presentation.
    \item Anticipate and prepare answers to potential questions.
    \item Practice the delivery for clarity and confidence.
\end{enumerate}

\section{Presentation Structure}
The presentation should follow this logical flow:

\subsection{Slide Outline}
\begin{enumerate}
    \item Title Slide: Thesis title, name, supervisor, and institution.
    \item Introduction: Problem, goals, and hypothesis.
    \item Results
    \item Background:
        \begin{itemize}
            \item Long-tailed datasets.
            \item Model architectures (ResNet-50, MobileNetV2, ConvNext-Base, ViT-B/16).
            \item Loss functions and how they address class imbalance.
        \end{itemize}
    \item Methodology:
        \begin{itemize}
            \item Dataset preparation (CIFAR-100 and CIFAR-100-LT).
            \item Model and loss function selection.
            \item Training and evaluation strategy.
        \end{itemize}
    \item Results:
        \begin{itemize}
            \item Performance comparisons (Top-1 accuracy, head, middle, tail-classes).
            \item Tail-class performance.
            \item Key takeaways.
        \end{itemize}
    \item Discussion:
        \begin{itemize}
            \item Interpretations and insights.
            \item Challenges and limitations.
        \end{itemize}
    \item Conclusion:
        \begin{itemize}
            \item Summary of contributions.
            \item Suggestions for future work.
        \end{itemize}
\end{enumerate}

\section{Core Topics to Review}
\subsection{Theoretical Topics}
\begin{itemize}
    \item \textbf{Convolutional Neural Networks:}
        \begin{itemize}
            \item Residual Networks.
            \item ResNet-50.
            \item MobileNetV2.
            \item ConvNext-Base
        \end{itemize}
    \item \textbf{Vision Transformers:}
        \begin{itemize}
            \item ViT-B/16.
            \item Self-attention mechanism
        \end{itemize}
    \item \textbf{Loss Functions:}
        \begin{itemize}
            \item Softmax activation function.
            \item Cross-Entropy Loss.
            \item Weighted Cross-Entropy Loss.
            \item Focal Loss.
            \item Class-Balanced Loss.
            \item Balanced Softmax.
            \item Equalization Loss.
            \item LDAM Loss.
        \end{itemize}
\end{itemize}

\subsection{Dataset Preparation}
\begin{itemize}
    \item Class distributions in CIFAR-100-LT.
    \item Data augmentation techniques and their impact.
\end{itemize}

\section{Pitfalls}
\begin{itemize}
    \item Optimizer (SGD, Adam, RMSProp, etc.)
    \item Transfer Learning.
    \item Fine-Tuning
    \item Data Augmentation.
    \item 
\end{itemize}

\section{Mock Questions and Answers}
\subsection{Potential Questions}
\begin{itemize}
    \item Why did you choose this topic?
    \item What are the main contributions of your thesis?
    \item How did you create the long-tailed dataset?
    \item Why did you choose these specific loss functions and models?
    \item What challenges did you encounter and how were they resolved?
    \item What are the limitations of your findings?
    \item How does your work advance the field?
\end{itemize}

\subsection{Prepared Answers}
\begin{itemize}
    \item \textbf{Why this topic?} Class imbalance is a critical issue in real-world datasets, and this work explores practical solutions.
    \item \textbf{Main Contributions:} Insights into loss function and architecture interplay for long-tailed image classification.
    \item \textbf{Dataset:} The CIFAR-100-LT dataset was derived using exponential decay to simulate real-world imbalances.
    \item \textbf{Challenges:} Handling transformer underperformance on small datasets and ensuring balanced evaluation metrics.
\end{itemize}

\section{Practice and Delivery Tips}
\begin{itemize}
    \item Use concise and clear language in your explanations.
    \item Highlight the relevance of your results to practical problems.
    \item Maintain a calm demeanor and take a moment to think before answering questions.
    \item Rehearse the presentation.
\end{itemize}

\section{Ideas}
\note{In this section I will write ideas as they come to me.}

\begin{itemize}
    \item 
\end{itemize}

\section{Practicalities}
\begin{itemize}
    \item Make slides interesting and pretty.
    \item Print slides.
    \item Bring a printed version of the thesis.
\end{itemize}


\end{document}