% Chapter 3: Methodology

This chapter describes the methods and approaches used in the experiments. This includes dataset preparation, models, loss functions, etc.

\section{Overview of Methodology Approach}
A high-level description of the approach to tackling the long-tailed dataset problem, including an explanation of the overall strategy, 
such as balancing techniques, model selection, and any specific objectives that guide the methodology.

\section{Algorithm Selection and Rationale}
Description of the algorithms or model architectures chosen, such as ConvNeXt or MobileNetV2, and why they are appropriate for long-tailed learning.
Discussion of the strengths and limitations of these models in addressing the challenges posed by imbalanced data.

\section{Long-tailed Learning Techniques}
Description of the specific methods used to address class imbalance, such as data sampling (e.g., oversampling/undersampling), class re-weighting, or advanced approaches like LDAM or DRW. 
Justification for selecting these techniques, potentially referencing prior research (e.g., from Deep Long-Tailed Learning: A Survey by Zhang et al.).

\subsection{Arguments}
The reason the focus has been on class-sensistive learning form the re-balancing methods, is that re-sampling comes with challenges regarding knowing the dataset prior to training the model. we want to find the best method for classifying the moth dataset, on which we have no ability to test. TODO: this might not be the case. 

\section{Loss Functions}
Explanation of the different loss functions explored, such as cross-entropy, focal loss, LDAM loss, etc., and their relevance for long-tailed learning.
Rationale for each loss function's inclusion, focusing on its expected benefits for imbalanced classes and how it addresses the bias toward majority classes.

\section{Data Imbalance Handling Strategies}
Detailed explanation of the techniques for creating and handling an imbalanced dataset, such as generating imbalanced training and test sets.
Any adjustments to the data pipeline to ensure that class distributions are maintained or specifically structured, as needed for the experiments.

\section{Evaluation Strategies}
Justification for the metrics and evaluation approach, such as using weighted or macro F1 scores.
Explanation of how you plan to assess performance across different class groups (e.g., head, middle, tail) to capture the model’s performance on minority classes.

\section{implementation Details}
Brief technical explanations of any unique or customized methods implemented in code, especially if they differ from standard practices.
Examples of changes made to existing algorithms or functions to adapt them for the long-tailed learning problem.

