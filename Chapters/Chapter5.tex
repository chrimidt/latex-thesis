% Chapter 5: Results and Analysis

Presentation of your findings, with tables, charts, and explanations for each tested method's performance.

Brief overview of the chapter’s purpose.
Recap the evaluation goals (e.g., assessing model performance across head, middle, and tail classes, and comparing methods).

\section{Overall Results}

Present the aggregate performance of all tested models and methods.
Use tables or charts to summarize key results (e.g., overall accuracy, F1 scores).
Highlight trends or notable observations across the methods.

\section{Head, Middle, and Tail Class Performance}
Break down the performance into head, middle, and tail class groups.
Include visualizations (e.g., bar plots or line graphs) showing metrics like F1 score or accuracy for each group.
Discuss how well the methods balance performance across these groups, particularly focusing on tail classes.

\section{Comparison of Loss Functions}
Analyze how different loss functions (e.g., cross-entropy, focal loss, LDAM) impact performance.
Use visual aids to compare results (e.g., per-class performance or confusion matrices).
Discuss trade-offs, strengths, and weaknesses of each loss function.

\section{Qualitative Results}
Optional.

Provide examples of correctly and incorrectly classified samples, especially for tail classes.
Include visualizations or images of difficult cases to highlight challenges in tail-class prediction.

% \section{Ablation Studies}
% Optional.

% If applicable, evaluate the impact of individual components or configurations (e.g., re-weighting, sampling).
% Discuss how removing or altering specific aspects affects performance.

\section{Summary and Discussion}
Recap the key findings, such as which methods or loss functions performed best and why.
Connect these findings to the thesis objectives and broader implications for long-tailed learning.