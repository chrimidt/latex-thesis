% Chapter 1: Introduction

% Length: aim for 2-3 pages.

\label{sec:introduction}
This thesis focuses on the problem with long-tailed datasets. The problem with training a deep learning model on long-tailed datasets is that the model will effectively 
the data from the classes with most samples, and not the classes with few samples. The finsihed model will then not recognize an input from the tail classes. Most real-world
datasets follows a long-tailed structure, hence the need for a reliable method to detect examples of tail-class data. The aim of this thesis is to try out some of the methods tackling the long-tailed problem for deep learning described in the paper \textit{Deep Long-Tailed Learning: A Survey} by Zhang et al.\cite{zhang2023deep} to find a method for long-tailed learning that works on a specific long-tailed dataset of images of moths taken around equator. The goal of the moth dataset is to identify species.

\section{Problem Definition}
Define the problem formally, including key terms like "head classes" and "tail classes."
Provide an example or visualization of a long-tailed dataset.
Connect the problem definition to the moth dataset.

\subsection{Goals of this thesis}
Outline the goals of the thesis, emphasizing optimizing performance on tail classes.
Mention how these goals contribute to the field of long-tailed learning.

\subsubsection{Hypothesis}


\subsection{Approach}
Summarize the approach to achieve goals, such as implementing and comparing methods from Zhang et al.’s survey.
% Highlight any novel aspects of the work (e.g., unique dataset, evaluation framework, or combination of methods).

\subsection{Scope of this thesis}
Specify the scope:
Focus is on image classification.
Methods are tested on a specific dataset.
The evaluation is limited to certain metrics.

\section{Motivation}
Discuss why the problem is significant, including real-world implications.
Mention the importance of biodiversity studies or the challenges of species identification with limited samples.
Discuss broader impacts, such as how solving long-tailed learning problems can benefit other fields.


\section{Reading Guide}
Mention what each chapter will cover and how they relate to each other.
% Example: "Chapter 2 provides the theoretical background for long-tailed learning and deep learning methodologies. Chapter 3 describes the methodology used in this thesis, including dataset preparation and implementation details. Chapters 4 and 5 present the experimental results and analysis, followed by conclusions and suggestions for future work in Chapter 6."

\section{Related Work}
A section that describes the work related to this thesis. 
