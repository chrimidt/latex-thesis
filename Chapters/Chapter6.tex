% Chapter 6: Conclusion and Future Work
This section revisits the goals outlined in Section~\ref{sec:goals}, summarizes the main findings from the experiments, and suggests directions for future work.

\section{Revisiting the Goals of the Thesis}
In revisiting the goals of this thesis, it has become clear that the investigation of the deep long-tailed learning technique Class Re-Balancing has yielded valuable insight. The goals and findings are presented in the following.

\noindent\textbf{Investigate the efficacy of long-tailed learning methods by assessing their performance on tail classes without sacrificing accuracy on head classes.}
The findings included that performance on tail classes often comes with a trade-off in performance on head classes. Similarly, adequate performances on all classes was often attributed to the performance on head classes and not inherently on tail classes. Nevertheless, Balanced Softmax Loss convincingly overcame this challenge across CNN-based architectures. This aligns with the goal of investigating the ability of deep long-tailed learning techniques to maintain overall accuracy. 

\noindent \textbf{Understand how model design affects the performance of deep long-tailed learning methods.}
The evaluations of different model architectures, namely the CNN-based and Vision Transformers, in combination with different class re-balancing techniques has revealed that model design has an effect on the performance of long-tailed learning. Specifically, ConvNeXt-Base showed robustness  

\noindent \textbf{Provide a comprehensive insight in comparisons that inform the choice of methods for long-tailed distributions.}
These findings contribute to a more comprehensive understanding of what to consider when training a deep network with long-tailed data, and, in doing so, serves as a guide to deep learning practitioners seeking more informed choices when encountered with a long-tailed problem.

\section{Summary of Main Findings}
The experiments conducted in this thesis underscore the complexity of deep long-tailed learning and highlight the importance of carefully combining model architectures, loss functions, and training configurations. 

Among the evaluated loss designs, the Balanced Softmax Loss consistently provided the best performance on minority (tail) classes across multiple CNN backbones. By re-scaling logits according to class frequencies, Balanced Softmax effectively mitigates the natural bias toward majority (head) classes. Although achieving comparable performances on head classes, the results indicate that this improvement sometimes comes at the expense of head-class accuracy, reflecting a broader trade-off in optimizing both head and tail classes simultaneously. 

% Among the evaluated approaches, the Balanced Softmax Loss consistently achieved the best performance on tail classes, while maintaining a comparable performance on other categories across multiple CNN backbones. In contrast, Class-Balanced Loss underperformed across all models and categories, suggesting a fault in implementation \todo{investigate}.

Equalization Loss underperformed across all models with the exception of ConvNext-Base, indicating a strong correlation with model robustnes, dataset size, and down-weighting of gradients. Likewise, LDAM loss, tested without deferred re-weighting (DRW), did not match the consistency of Balanced Softmax, reaffirming that additional sampling strategies might be needed to fully realize LDAM's potential.

The ViT-B/16 architecture showed a pronounced performance gap compared not only to CNN-based architectures but also to its own reported benchmarks. Transformers often require large-scale and well-balanced datasets, as well as careful configuration of hyperparameters and training schemes, to perform at their peak. By contrast, CNN backbones generally demonstrated stronger performance and robustness under long-tailed conditions.

\noindent Overall, these results illustrate that improvements in tail-class accuracy often come with compromises in head-class performance and overall accuracy. This underscores the importance of carefully balancing strategies in long-tailed learning, including backbone architecture, loss function design, dataset size, parameter fine-tuning, and optimization strategies.

\section{Future Work}
\paragraph{Scaling to Larger Datasets}
Future investigation would benefit from evaluating experiments on large-scale, long-tailed benchmarks, such as ImageNet-LT or iNaturalist, to gain more insight into the performance of models and long-tailed techniques. The greater variety in tail classes and more realistic Pareto-like distributions could reveal whether the observed trends hold at scale, especially in the case of transformers, which typically require larger datasets and additional hyperparameter tuning. Training ViT-B/16 on a large-scal dataset with fine-tuning schedules tailored to preserve lower-layer feature representations may unlock its full potential and close the performance gap observed here.

Additionally, training ViT-B/16 on larger datasets while carefully considering its configurations could prove its full potential. Furthermore, evaluating the LDAM including the DRW and, potentially, combined with other loss functions on CIFAR-100-LT could reveal under which conditions tail-class performance is maximized. Other techniques, such as re-sampling in combination with class re-balancing, could be explored. Finally, combining strong performances and ensemble learning can reveal extraordinary performances.

\paragraph{Refining Class-Sensitive Learning with Other Long-Tailed Methods}
Other studies have shown that combining LDAM with deferred re-weighting (DRW) on CIFAR-100-LT enchance performance on tail classes \cite{cao2019learningimbalanceddatasetslabeldistributionaware,menon2021longtaillearninglogitadjustment}. Investigating how LDAM-DRW interacts with other re-balancing strategies or advanced optimizers could pinpoint conditions under which tail-class performance is maximized.

\paragraph{Freezing Model Parameters}
Given the trade-offs of fully fine-tuning pretrained models, exploring partial freezing or gradual unfreezing of layers might better retain universal features while still adapting effectively to the distribution of minority classes. This layered approach could mitigate overfitting risks, particularly for smaller long-tailed datasets like CIFAR-100-LT.

\paragraph{Combining Approaches and Ensemble Learning}
In practice, a single technique may not suffice to handle highly skewed class distributions. Future work could explore combining the strongest methods, such as Balanced Softmax, and integrating them into ensemble architectures. Ensemble learning may help strike an optimal balance between tail-class gains and overall classification accuracy.


% \section{Summary}
% Balanced Softmax Loss achieved highest accuracy on minority classes across all CNN architectures by a large margin, highlighting the effectiveness of logit shifting compared to re-weighting and re-margining techniques. However, adjusting logits comes with a trade-off in majority classes, where weighted cross-entropy loss designs consistently outperforms.

% Overall, findings of these experiments emphasize the importance of aligning loss functions with both dataset characteristics and model architectures to address the challenges of deep long-tailed learning. They also highlight the need for a nuanced approach that balances improvements in tail-class accuracy with minimal trade-offs in head-class and overall performance.

