% Chapter 4: Experimental Setup

This chapter focuses on the on the implementation details of the experiments conducted in this thesis. Here, the specifics of the training configurations are described. 

\section{Dataset Specifications}
Details about the dataset(s) used, including size, source, and preprocessing steps.
Description of class imbalance characteristics and the train/validation/test splits.

\section{Data Preprocessing}
Any transformations, augmentations, or normalization applied to the dataset before feeding it to the model.
Information on how the class imbalance is handled (re-sampling or synthetic data generation).

\section{Model Architecture Settings}
Description of the models used, including any specific architecture choices, hyperparameters, or modifications.
Brief details on why these models were chosen.

\section{Training Configurations}
Hyperparameters, such as batch size, learning rate, optimizer type, and regularization techniques (dropout, weight decay).
Any specific settings for handling long-tailed data, such as DRW.

\section{Evaluation Metrics}
How?

\section{Hardware and Software Configurations}
Hardware details.
Software environment, including the versions of libraries and frameworks.

\section{Reproducibility Considerations}
Steps taken to ensure that results can be reproduced, such as random seed initialization and details on dataset versions.
Scripts, configurations, or instructions for reproducing experiments.
