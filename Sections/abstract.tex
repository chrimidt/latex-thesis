Long-tailed datasets, where a few classes dominate with abundant samples while many classes have sparse representation, pose significant challenges for traditional training methods. These imbalances often lead to models that perform well on majority classes but struggle to recognize or generalize to minority (tail) classes.

% This thesis focuses on evaluating and implementing re-weighting methods for long-tailed learning, as outlined in the survey Deep Long-Tailed Learning: A Survey by Zhang et al. The methodologies explored include advanced sampling strategies, re-weighted loss functions, and modifications to deep learning architectures tailored to imbalanced data.

% A unique application of these methods is demonstrated on a custom dataset of moth images collected near the equator, where the goal is accurate species identification. Through a series of experiments, the thesis investigates how different approaches to long-tailed learning impact model performance across head, middle, and tail classes.

% The findings contribute to understanding the efficacy of these methods and provide insights into best practices for handling real-world long-tailed datasets.